\section{Zaključak}
Raspberry pi zbog svojih malih dimenzija, relativno dobrih performansi te izuzetno lake prenosivosti se pokazao kao
odličan izbor za nadzornu kameru, također širok izbor dodatnih modula nudi mogućnost velikog broja potencijalnih
nadogradnji, primjerice mikrofon ili zvučnik.
\paraBreak
Sam proces živog prijenosa s druge strane, se zbog nužnog koraka kompresije videa odnosno enkodiranja dokazao
kao jako velik izazov koji zahtjeva puno istraživanja i generalnog znanja o temi. Biblioteka korištena za enkodiranje
iako omogućuje cijeli proces jako je nezgodna za korištenje zbog sirovog sučelja izloženog korisniku te zahtjeva
puno dodatnog truda oko razumijevanja ogromnog broja struktura i funkcija koje su dio iste.
\\
Unatoč tome rezultati samog enkodiranja govore za sebe, kao što je vidljivo na slici \ref{graf:packet_size} 
drastično smanjenje totalne veličine uz minimalne gubitke  kvalitete slike je impresivno.
\\
Ovaj rad je samo ogrebao površinu mogućnosti enkodiranja i ostavlja puno prostora za poboljšanja, zbog velikog izbora
kombinacija kodeka i kontejnera te prividno beskonačno postavki istih, uz daljnje eksperimentiranje moguće je
postići još bolje rezultate.
\\
Osim poboljšanja u području enkodiranja može se znatno poboljšati integracija, primjerice transkoder ili poslužitelj 
bi na raznim koracima mogli slati zahtjeve na poveznice definirane od strane korisnika, takozvani \foreign{web hook}-ovi.
\paraBreak
Konačno poslužitelj koji je implementiran kao potpuno neovisan dio sustava kao takav omogućava laku integraciju sa
vanjskim sustavima koristeći ustanovljene načine komunikacije kao što su TCP i HTTP što je demonstrirano sa
implementiranim web klijentom.
\\
Također zbog takve neovisne arhitekture lako je moguće imati veći broj kamera spojene na isti centralni poslužitelj 
koji se ponaša kao središte te dalje putem HTTP-a prenosi dobivene informacije potencijalnim klijentima što je za jedan 
nadzorni sustav ključna komponenta. U ovom dijelu arhitekture također ima mjesta za daljnji razvoj, primjerice jedna on 
originalnih ideja ovog rada je bila da poslužitelj prosljeđuje dobivene slike od transkodera nekom trećem djelu sustava koji
bi detektirao potencijalne točke interesa na slici te ih označio, u suštini prepoznavanje lica osoba ili nešto slično.
