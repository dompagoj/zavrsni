\section{Zaključak}
Raspberry pi zbog svojih malih dimenzija, relativno dobrih performansi te izuzetno lake prenosivosti se pokazao kao
odličan izbor za nadzornu kameru, također širok izbor dodatnih modula nudi mogućnost velikog broja potencijalnih
nadogradnji, primjerice mikrofon ili zvučnik.
\paraBreak
Sam proces živog prijenosa s druge strane, se zbog nužnog koraka kompresije videa odnosno enkodiranja dokazao
kao jako velik izazov koji zahtjeva puno istraživanja i generalnog znanja o temi. 
\\
Unatoč tome rezultati samog enkodiranja govore za sebe, drastično smanjenje totalne veličine uz minimalne gubitke 
kvalitete slike je impresivno.
\\
Ovaj rad je samo ogrebao površinu mogućnosti enkodiranja i ostavlja puno prostora za poboljšanja, zbog velikog izbora
kombinacija kodeka i kontejnera te prividno beskonačno postavki istih, uz daljnje eksperimentiranje moguće je
postići još bolje rezultate.
\paraBreak
Konačno poslužitelj koji je implementiran kao potpuno neovisan dio sustava kao takav omogućava laku integraciju sa
vanjskim sustavima koristeći ustanovljene načine komunikacije kao što su TCP i HTTP što je demonstrirano sa
implementiranim web klijentom.
\\
Također zbog takve neovisne arhitekture lako je moguće imati veći broj kamera spojene na isti poslužitelj koji se 
ponaša kao središte te dalje prenosi dobivene informacije potencijalnim klijentima što je za jedan nadzorni sustav 
ključna komponenta.
